\chapter{Introduction}

This report describes various facts used to select thesis statement “Development of verification environment for SHA-3 core using UVM” and follow up approach to realize development of solution for the problem stated. Here, verification environment for SHA-3 Cryptographic Core is proposed and developed using UVM methodology. Literature which is surveyed to come up for the thesis selection is outlined in Literature Survey section. SHA-3 is cryptographic hash function and used in many applications including data integrity, password matches, etc. Proposed verification uses concept of OOPs provided in SystemVerilog language with UVM industry standard methodology used in ASIC/SoC verification.

\section{Problem Statement}

This dissertation work is induced from need of development of UVM based verification environment for SHA-3 cryptographic core. Taking this need in consideration, dissertation work describes flow from specification extraction to development of verification environment.


\section{Motivation}

Following increasing logic complexity in digital ICs, verification requirements has also become complex. To address this issue SystemVerilog Hardware Verification Language (HVL) came into existence. SystemVerilog is comprehensive base language which supports OOPs and thus abstraction level in testbench. But, to practice reusability in our testbench environment, UVM methodology is developed later which has rich set of class library. \par

SHA-1 is most widely known and used hash function in several applications and protocols. Many successful attacks have been noted on SHA-1 which led NIST to move to SHA-2 after 2010 because of the weakness in previous hash algorithm. Although no successful attacks have been noted on SHA-2, development of another option is necessary as future perspective. NIST has selected a new cryptographic hash algorithm which is referred to as the Secure Hash Algorithm 3 (SHA-3) and is intended to complement the SHA-2 hash algorithms currently specified in Federal Information Processing Standard (FIPS) 180-3, Secure Hash Standard. The selected algorithm is intended to be suitable for use by the U.S. government as well as the private sector and is available royalty-free worldwide\cite{fips202}. \par

Combining above best practices, flow of development of verification environment for Keccak SHA-3 core using UVM is described in this report.

\section{Scope}

The thesis aims at applying hardware verification concept to the cryptographic core and check functional correctness of the core. Find out appropriate verification methodology and build verification environment according to it. Then integrating our developed environment with the design under test and verifying appropriate response in comparison to functional model while achieving complete code coverage. Scope of the dissertation work can also be extended to functional coverage and redefining hardware block by critical assessment of work done on Secure Hash Algorithms. Proposed environemt can be used to verify other cryptographic RTLs with minimun configuration changes exploiting reusability in UVM based environment.

\section{Objectives}

Work to be presented in this dissertation thesis aims at following standard verification flow used in industry to develop verification environment for the cryptographic core. \par

Critically analyzing the relevant literature of Secure Hash Algorithm and Verification techniques with UVM is the first step. The work will follow standard practices to develop verification environment for verification methodology which is Universal Verification Methodology. \par

One important aspect of hardware verification is code coverage, which will also be covered. \par

Scope of this work can also be extended to generate functional coverage. Concept of reusability can also be explored. And area vs. throughput consideration can be analyzed and studied for hardware design improvements and FPGA implementation. \par

\section{Research Methodology}

First step in the dissertation flow is to understand Design under Test (DUT) specification. Following feature extraction from the design, this is specification of our Keccak SHA-3 core. Interpretation of design features is used to generate test scenarios under which selected RTL will be operating. \par

It is then followed by testbench architecture development phase. This phase contains developing UVM testbench hierarchical components. Development and execution of environment is done on verification tool QuestaSim 10.0b by Mentor Graphics tool. Code coverage and functional coverage functionalities are also provided by Mentor Graphics Questa Advanced Simulator. \par

File format like .sv .svh .do .f and etc for required verification environment are created using emacs editor under Linux Ubuntu 15.04. And documentation for creating thesis report is done on Gummi Latex Editor tool.\par

Use of scripting language like Perl or TCL can also be studied to create Makefile or Do file and command line arguments to run tests and batch execution. \par

Scope of this dissertation work if extended to design and implementation of SHA-3 Keccak on FPGA then other tools Xilinx ISE, Timing Tool Editor Etc can be used to create RTL code, linting and synthesizing the RTL and producing technology view of the RTL. And proposed verification environment can be applied to verify functionality of RTL using exhausted number of testcases provided.

\section{Expected Outcome}

Expected outcome from this dissertation work include, UVM based verification environment integrated with SHA-3 core, verification environment component will be created and mapped to UVM library. Complete code coverage will be achieved by applying randomly generated testcases. Checks for different test scenario will be done. And appropriate documentation of prescribed work will be done.