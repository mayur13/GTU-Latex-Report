\chapter*{Conclusion}
\addcontentsline{toc}{chapter}{Conclusion}

Aim of the thesis work is to create verification environment for SHA-3 Crypto Processor Core in SystemVerilog Hardware Description and Verification Language using UVM 1.1 accellera standard methodology. UVM 1.1 library comes with QuestaSim 10.0b Simulation tool. The UVM based environmet contains standard UVM Verification Components, Driver, Monitor, Agen, Env etc, connected via virtual interface. Complexity of Keccak SHA-3 Crpto Core requires structural testbench concepts used in industry. UVM methodology is industry standard methodology evolved from OVM and VMM. With the use of HVL and appropriate methodology, to verify DUT, coverage management helps keep track of coverage result in terms of code coverage and functional coverage. Appropriate coverage closure enables achieve time-to-market and cost control. Coverage management being important aspect, in this work code coverage report has been created using test regression and 'coverage' UVC keeps track of functional coverage. Testcases have been developed considering test scenarios for appropriate functionalities. Copy of working environment is available online on Git www.github.com/mayur13/UVM-Verification-Environment.