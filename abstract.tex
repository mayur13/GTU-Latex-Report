\thispagestyle{plain}

\begin{center}
	\vspace{1cm}
		
	\bigsize{\textbf{Development of Verification Environment for SHA-3 Core using UVM}}
		
	\vspace{1.5cm}
		
			\normalsize
			By\\
			\textbf{Mayur Navinchandra Kubavat}\\
			\textbf{(131060752013)} \\
			Guided By\\
			\textbf{Mr.Ashish Prabhu}\\
			\textbf{Sr. Verification Engineer, LSI India R\&D}
			
	\bigsize\textbf\textit\underline{ABSTRACT}
\end{center}

As CMOS technology advances, complexity of ICs also increases. Moore’s law states that number of transistor in ICs doubles approximately every two years. Verifying working of these complex digital systems has become more important now, because any small deviation from specification in high density chips can lead to system failure and complete re-spin of the design cycle. Which then leads to increased cost and delay in time-to-market. Therefore SystemVerilog HVL was developed to create testbench for large digital design. SystemVerilog supports concepts of OOPs and provide higher level of abstraction in our testbench. To support reusability of verification components UVM is developed later which is derived from OVM and eRM. \par

This dissertation work is induced from necessity of development of UVM based verification environment. Proposed work in this thesis describes flow from specification extraction to development of verification environment and implements UVM based reusable verification environment. \par